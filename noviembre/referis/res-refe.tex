\documentclass [12pt]{letter}
\usepackage[utf8]{inputenc}%agregado
\usepackage{color}
\input epsf
\topmargin 0cm \textheight 20cm


\begin{document}

\signature{Gonzalo Maximiliano López}
\address{Dr. Gonzalo Maximiliano López\\
 INENCO\\ Universidad Nacional de Salta\\
 4400 Salta, Argentina}


\begin{letter}{Dr. Ricardo Castro Santis\\ Editor Journal of mathematical modeling of biological systems}


%\epsfysize=3truecm \epsffile{c:/juan/papeleo/templates/logo.eps}
%\vspace{-4cm}

\opening{Dear Editor: }

Together with this letter you will find the revised version our manuscript {\it
Modeling macroparasite infection dynamics} by Gonzalo M. López and Juan P. Aparicio.

Following the observations of both referees we carefully rewrote the manuscript. Major changed paragraph are in red, while many typos were corrected but not highlighted.

I hope this new version would be suitable for publication in the Journal of mathematical modeling of biological systems.

Thank you for your attention.

\closing{Sincerely,}
\thispagestyle{empty}

\newpage
REPLY to the referees





\textbf{Reviewer 1}

We thank the reviewer for his/her careful reading of our manuscript.
	
\begin{enumerate}
		\item Perform an in-depth review of the manuscript, including the writing
		in English.
		
		This new version of our manuscript was carefully edited, and the
		English was revised.
		
		\item The authors must use the Template of the Journal.
		
		Done.
		
		{\color{red}
		\item The original contributions need to be much better presented in the
		last paragraphs of section ``INTRODUCTION”. All improvements, if they are, and new results must
		be described in this paragraph.
		
		We included a paragraph following the reviewer advice.
		}
		
		\item $\lambda_0$ is not defined.
		
		Done.
		
		\item In Fig 2, Put the saddle-node bifurcation.
		
		Done.	
		
		\item Include a Theorem with hypothesis about when the systems (5-6) undergoes the
		saddle-node bifurcation.
		
		Done.
		
		\item In model (21) $F (m)$ is not defined.
		
		We included the definition of $F(m)$ in this new version.
		
		\item In heterogeneous model (21), bifurcation analysis is more complicated, however
		numerical tests by considering different values of $R_i$ can be considered, in order to
		better understand the dynamics of the model. A similar analysis can be found in
		B\"urger, R. , et al. ``Modelling the spatial-temporal progression of the 2009 A/H1N1 influenza pandemic in Chile.” Mathematical Biosciences \& Engineering, 2016, vol.	13, no1, p. 43.
		
		Done.
\end{enumerate}

\newpage	

\textbf{Reviewer 2}
	
We thank the reviewer for his/her carefull reading of our manuscript.	

\begin{enumerate}
	{\color{red}
	\item Title: I suggest replacing “Modelling macroparasitic diseases dynamics” by
	“Modeling macroparasite diseases dynamics”.
	
	Done.?
	}
	
	{\color{red}
	\item Abstract: if an abstract does not contain “we” or phrases like “in this paper”
	usually is well received. I suggest changing these expressions. Similarly, I also
	recommend mentioning the homogeneous and heterogeneous focus.
	}
	
	
	{\color{red}
	\item Introduction: I think that information is missing, for example, indicate:
	\begin{itemize}
	\item Problem to solve, principal objective or motivation.	
	\item Route of solution or response indicating the sections
	\end{itemize}
	}
	
	{\color{red}
	\item General framework: Here, I consider the authors should connect it with the
	objective or motivation of your proposal. Also, include comments about your
	previous work cited as [9].
	}	
	
	
	{\color{red}
	\item Subsection 3.1: In the first line, the authors mention that their model a is based on
	a model developed by Anderson and May, but they do not explain or mention the
	modifications applied to it for their proposal.
	}

	\item Equations: I suggest ending the equation with “.” or “,” to give continuity to the text
	and the reading.
	
	Done.
	
	\item I recommend including an informative figure or graph about the phenomenon or
	dynamics studied.
	
	Done.
	
	\item Equation (1): What does the function $\Gamma()$ represent? Can you say something about
	the fractions $k/(m+k)$ and $m/(m+k)$?
	
	Done.
	
	\item Page 5 in the second paragraph, I recommend mentioning what information is
	taken from reference [7]. Similarly, with reference [5], in the third paragraph.
	
	We included the reviewer advice in this new version.
	
	\item Replace “([9])” by “[9]”.
	
	Done.
	
	\item Page 6 in the text of equations (5)-(6), I suggest including a paragraph to conclude
	the homogeneous case of the proposed model. Maybe they need to explain a little
	more about it.
	
	We included a paragraph following the reviewer advice.
	
	\item Equilibria and basic reproduction number: I consider it important to define or
	introduce what represents the equilibrium and the $R_0$ basic reproduction number in
	the model proposed. Can the authors say something about the expression
	$R_0 \lambda_0 \alpha \rho /(\mu_h + \mu_p)$?
	
	We included a paragraph following the reviewer advice.
	
	\item Page 7: Sensitivity analysis, Do they have any references for this analysis? Could
	the authors show some graphs to represent these indices?
	
	We included a sensitivity analysis in this new version.
	
	\item Page 8: Expand equation (19) and terms $A=(\mu_h + \mu_p)m/R_0$ and $B$ for to have a
	better read.
	
	Done.
	
	\item Equation (21), I recommend explain and extend the meaning of the terms $m_i, \beta_i,
	\rho_i$...etc. Maybe also include an explanation of these equations in terms of the
	phenomenon or dynamic studied.
	
	We included a paragraph following the reviewer advice.
	
	\item Subsection 4.1 What is the difference between the indexes “ i” and “j”?
	
	We clarify the difference between the indices in this new version.
	
	
	\item Page 11. Expand equations of $R_0^i$ and $m_i$ . Replacing by
	“where we define the basic reproductive number of each subpopulation $H_i$ by
	which is the number of adult females that are born of a adult female from a host in
	subpopulation $H_i$ in the absence the effects of density-dependence and the mating
	probability. Note what for a large $N$ value the reproduction number for each $H_i$ is
	given by"
	
	Done.
	
	{\color{red}
	\item Discussion and Conclusions: I suggest including a summary of the results, with
	analysis and conclusions. Also, I recommend including an example of some
	scenarios, perhaps, varying parameters, these to evidence the need for future
	works. I feel that you could comment more here.
	}
	
	
	
\end{enumerate}


\end{letter}
\end{document}	
	
	
	
	2. 
	
	Done.
	
	3. 
	
	
	
	4. λ0 is not defined.
	
	done.
	
	5. In Fig 2, Put the saddle-node bifurcation.
	
	done.
	
	6. Include a Theorem with hypothesis about when the systems (5-6) undergoes the
	saddle-node bifurcation.
	
	Done.
	
	7. In model (21) F (m) is not defined.
	
	We included the definition of F(m) in this new version.
	
	8. In heterogeneous model (21), bifurcation analysis is more
	complicated, however
	numerical tests by considering different values of Ri can be
	considered, in order to
	better understand the dynamics of the model. A similar analysis can be found in
	Bürger, R. , et al. ”Modelling the spatial-temporal progression of
	the 2009 A/H1N1
	influenza pandemic in Chile.” Mathematical Biosciences  Engineering, 2016, vol.
	13, no1, p. 43.
	
	viste este trabajo?
	
	
	
	
	\item \textit{Macroparasites usually present over-dispersed distributions, that is, a few hosts harbour many parasites, while the remainder of the hosts are virtually parasite free (see, for example,[6][20])
	\\A causa de super-dispersão é essa? Essas referências [6] e [21] falam disso? Mostram em que parte.}

In this paper we present some alternatives to the negative binomial distribution for the description of the commonly overdispersion observed in samples of macroparasites. References [6] and [21] provide examples of such samples. Causes of overdispersion are not discussed in our paper neither in the referenced cited above, however we included a new reference where the possible causes of overdispersion are presented ([10]).


	
	\item \textit{ When $\pi<0$, we have a zero-deflated distribution.\\ 
	Pode ser negativo? $\pi$ não é uma proporção?}
	
	It is possible to take $\pi<0$ (zero-deflated model),
	provided that $\pi + (1-\pi)p_0\geq 0$ , that is $\pi\geq \frac{-p_0}{1-p_0}$

	\item \textit{ Em (1) os únicos parâmetros que aparecem são $\pi$ e $\theta$. Como surgiu $\mu$ e $\sigma^2$?} 
	
	In equation (1) it is mentioned that $\theta$ is a vector of parameters, so it is formed by several parameters, such as $\mu$ (mean) and $\sigma^2$ (variance)

	\item \textit{In Table 1\\
		O que são essas letras? Faltou defini-las.}
	
	The letters label the  different samples of the analyzed data. Its definition has already been added to the text.
	
	\item \textit{Por que foi realizado teste qui-quadrado?}
	
	We used the chi-square test as asimple goodness-of-fit test. This was clarified in the new version of our manuscript.
	
	\item \textit{In the section Discussion and Conclusions. Sugestão: apresentar um estudo de simulação.}

	The simulation study has been added as an appendix section of the article.

\end{itemize}

\textbf{Minor issues:} We thank the reviewer for his/her carefull reading of our manuscript. All the minor issues were corrected and added to the text in this new version.


\textbf{Reviewer 2}

\textbf{Major issues:}

\begin{itemize}
	\item \textit{``where we denote by $\theta$ to the vector of parameters of the associated distribution $p$ and then''\\
	check the concept because $p$ is a probability not a distribution}

We clarified the meaning of the probability $p$ as suggested by the referee. 
	
	
	
	\item  \textit{``which are not familiar with programming and the numerical implementation of algorithms''
		\\This is not a good defense because the
		computational area has grown considerably in
		the last few years. The justification would be for
		the model to be simpler, with estimates easier to
		obtain}
	
	Referee is absolutely right about this, we improved this justification in the new version.

	
\end{itemize}

\textbf{Minor issues:} We thank the reviewer for his/her carefull reading of our manuscript. All the minor issues were corrected and added to the text in this new version.


\end{letter}
\end{document}